%!TEX root = ../../book_ML.tex
\chapter{Cơ sở lý thuyết}
\label{cha:chap2}
% \index{principal component analysis}
% \index{PCA -- \textit{xem} principle component analysis}
% \index{PCA}

% \index{phân tích thành phần chính -- principle component analysis}
% \index{principle component analysis -- phân tích thành phần chính}
% \index{PCA}
\section{Tổng quan về nhận diện khuôn mặt}

Nhận diện khuôn mặt (Face recogintion) đang được ứng dụng trong nhiều lĩnh vực.
Hệ thống nhận dạng khuôn mặt là một ứng dụng cho phép máy tính tự động xác định
hoặc nhận dạng một người nào đó từ một bức hình ảnh kỹ thuật số hoặc một khung
hình.

Nhận diện khuôn mặt là một bài toán phức tạp, đòi hỏi cần phải xử lý một
loạt các vấn đề .

Mỗi khuôn mặt đều có nhưng điểm mốc, những phần lồi lõm, hình dáng của các
bộ phận trên khuôn mặt như mắt, mũi, miệng,... Các hệ thống nhận diện định
nghĩa những điểm này là những điểm nút, và mỗi khuôn mặt có khoảng 80 nút như thế


\section{Tìm hiểu về OpenCV}
\section{Mô hình mạng neural tích chập (CNN - Convolutional neural network)}
\section{Máy dò khuôn mặt (Face detector)}
\section{Các kĩ thuật làm giàu dữ liệu (Data agumentation)}
\section{Mô hình học sâu được huấn luyện trước (Pre-train model)}
\subsection{Sử dụng mô hình được huấn luyện trước}
\subsection{Giới thiệu Facenet}
\subsection{Giới thiệu Mạng InceptionResnetV1}
\subsection{Kĩ thuật đánh giá bộ ba (Triplet loss)}

